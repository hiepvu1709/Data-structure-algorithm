    Tu dong tang kich thuoc cua vector khi nhap phan tu vao
    
    push_back(): them phan tu vao cuoi vector
    pop_back(): xoa phan tu cuoi vector

    v.front(): tra ve phan tu dau vector
    v.back(): tra ve phan tu cuoi cung vector

    size()

    truy cap thong qua chi so 

    Duyet for / foreach / iterator (con tro) / auto
        %     for (int i = 0; i< v.size(); i++){
        %     cout << v[i] << endl;
        % }

        %     for(int x : v){
        %     cout << x << endl;
        % }

        %     for (vector<int>::iterator it = v.begin(); it != v.end(); it++){
        %     cout << *it << endl;
        % }

        %     for (auto it = v.begin(); it != v.end(); it++){
        %     cout << *it << endl;
        % }

    v.end(): kp la phan tu cuoi cua vector v, no o sau phan tu cuoi cung

    v[2] ~ *(v.begin() + 2)

    Co hai cach nhap gia tri cho mang vector:
        + Cach 1: co the nhap truc tiep trong for nhu mang binh thuong
            vector<int> v(n);  voi n la so luong phan tu 
            for(int i = 0; i <n ; i++){
                cin >> v[i];
            }
        + Cach 2: thong qua 1 bien tam
            vector<int> v;
            for(int i = 0; i <n ; i++){
                int x;
                cin >> x;
                v.push_back(x);
            }
    
    Khai bao mang vector co gia tri duoc chi dinh:  vector<int> v(n, 100);

    Co thay thay doi kich thuoc cua vector khi da co cac gia tri ben trong
        v.resize(value);

    Trong hàm xóa các phần tử, lý do mà r + 1 là bởi vì nó sẽ xóa được cả phần tử có vị trí r